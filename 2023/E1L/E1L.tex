\chapter{Economics in One Lesson}{Henry Hazlitt}
\section{Preface}
\begin{outline}
\1 Goal of the book: address common fallacies in economics.
\1 The book is an exposition, not a proposition.
	\2 All the described ideas are old and well-known, so their exact origin isn't important.
	\2 Misconceptions are also part of the public consciousness, and hence the author won't target specific authors' flaws (Marx etc.).
\end{outline}

\section{The Lesson}
\begin{outline}
\1 Economic fallacies are spawned in two ways:
	\2 Only considering effects on one group of the population.
	\2 Only considering effects in the short term.
	
\1 The lesson:
\begin{quote}
	\emph{The art of economics consists in looking not merely at the immediate but at the longer effects of any act or policy; it consists in tracing the consequences of that policy not merely for one group but for all groups.}
\end{quote}
	\2 Classical economists used to make the opposite error (only look at long term, not short term, which is also important).	
	\2 Demagogues sell their ideas by pointing to short-term single-group benefits, without the full story.
\end{outline}

\section{The Broken Window}
\begin{outline}
\1 The \emph{broken-window fallacy} goes against the lesson:
	\2 A bakery is vandalised with a brick through the window. A new window costs $x$.
	\2 This seems good: repair window $\Rightarrow$ the glass-maker earns $x$ $\Rightarrow$ glass-maker can spend $x$ at various sellers $\Rightarrow$ they can spend $x$ at various sellers $\Rightarrow$ ... $\Rightarrow$ almost everyone in the community benefits from that $x$.	
	\2 ... but the baker has lost $x$ $\Rightarrow$ the tailor who was going to make his suit didn't gain $x$ $\Rightarrow$ he can't spend that $x$ in the community. The baker could've had a window and a suit, now he has a window and no suit (or a suit and no window).

\1 No net new jobs have been created: 
	\2 The glass-maker is now employed.
	\2 The tailor is no longer employed.
\end{outline}

\section{The Blessings of Destruction}
\begin{outline}
\1 The broken-window fallacy is used in multiple ways to justify war.

\1 "Post-war economies have high demand, so they grow a lot."
	\2 Conflates "demand" and "need". I was planning to buy a car, not a house; my house got bombed and I need one.
	\2 The economy doesn't \emph{grow}, but it \emph{shifts}. Demand is \emph{diverted}, not \emph{created}.

\1 "Bombed countries have an advantage because they can now build the newest factories."
	\2 If that was true, why doesn't everyone just demolish their factories?
	\2 There is an economically theoretical optimum to replace equipment. It depends on 
	\begin{enumerate}
		\item depreciation (What's the remaining output of the factory?)
		\item obsolescence (Do there exist better factories?)
		\item available capital (Can you afford a new factory?)
	\end{enumerate}
	and if you're bombed before that optimum, you have net loss.

\1 Aside: suppliers are demanders.
	\2 In an economy with separation of concerns and trade, the only way to obtain a product/service is to offer another one. Low supply means the suppliers have no demands.
\end{outline}

\section{Public Works Means Taxes}
\begin{outline}
----------------------------------------

\1 Taxes are only justifiable from a free-market perspective if the taxpayers themselves would have spent the taxes on the same things as the government.
	
\1 Government inefficiency is baked into the goal of "job creation". 
	\2 Getting the project done inefficiently $\Rightarrow$ more working hours $\Rightarrow$ more labour.

\1 Inflation is taxation in disguise.
\end{outline}

\begin{outline}
\1 Businesses foot 100 cents of every dollar they lose, but keep only 50 cents of every dollar they earn.
	\2 Taxes disincentivise risk-taking, e.g.\ investment, because you foot all of the loses and keep only some of the gains.

\1 With a 60\% income tax, you spend 7 months per year solely to pay the government. Not a penny for you.

\end{outline}

\section{Credit Diverts Production}

\section{The Curse of Machinery}
\begin{outline}
\1 Still relevant in 2023 with improving robots and AI.
\end{outline}

\section{Spread-the-Work Schemes}
\begin{outline}
\1 
\end{outline}

\section{Disbanding Troops and Bureaucrats}
\begin{outline}
\1 In this chapter: is it a good idea to cut certain taxpayer-funded jobs?

\1 Does disbanding a drafted army after war cause high unemployment?
	\2 In the short term, yes.
	\2 In the long term, no.
		\3 The soldiers were supported by government funding.
		\3 The government got that funding from taxes.
		\3 Reduce the heightened wartime taxes,\footnote{It is the budgetary responsibility of the government to \emph{not} find new ways to spend taxes that have become obsolete. Otherwise, taxes can never lower.} and citizens have more money to spend.
		
\1 Disbanding an excess of soldiers increases total wealth.
	\2 After war, those soldiers are being paid without providing value.
	\2 If they are employed, and citizens spend all their wartime taxes, then citizens now get actual value for their money!
	\2 The same applies to an excess of bureaucrats.
	
\1 Won't the economy shrink -- less is purchased -- when taxpayer-funded jobs are cut?
	\2 The useless bureaucrats lose wage, the citizens gain it due to tax decrease.
	\2 Is it net $0$? No, it's net positive: the bureaucrats will become privately employed and actually provide a service for that same money.
	\2 The argument of "Won't the economy shrink" can be applied to thieves: thieves steal your money and support businesses with it, \emph{but} give you nothing in return. Useless bureaucrats are like thieves.
	\2 In fact, many useless bureaucrats are \emph{worse than thieves}, because they also make regulation to harm the free market.
\end{outline}

\section{The Fetish of Full Employment}
\begin{outline}
\1 \emph{Full employment}: having $0$ unemployed citizens. 
	\2 This is actually very easy: give them bogus jobs, or force them to work.
		\3 In 3rd-world countries, there is full employment.
		\3 In a war, there is full employment.
	\2 What we actually want is \emph{full productivity}. The one-directional law holds:
	\begin{equation}
		\text{full productivity} \Rightarrow  \text{full employment}
	\end{equation}
	... but not the reverse. Employment is the means, not the end.
	
\1 Full employment signifies lack of innovation.
	\2 The whole point of innovation is to save labour: the wheel, machinery, ...
	\2 Innovation necessarily makes some jobs obsolete, causing short-term unemployment.
\end{outline}

\section{Who's "protected" by tariffs?}
\begin{outline}
\1 \emph{Tariff}: fee paid to the government of a country by its citizens for buying a certain product in a certain country.
	\2 E.g.: American sweaters cost \$30, British sweaters cost \$25, then a \$5 tariff on British sweaters incentivises Americans to buy in America.
		\3 Assume the quality is exactly the same. If British sweaters are better, raise the tariff by their additional value.
	\2 Two tariff situations to consider:
		\3 Should we revoke an existing tariff for an existing industry?
		\3 Should we impose a new tariff to allow a non-existent industry to form?
		
\1 Americans lose purchasing power. 
	\2 Without the tariff, a sweater of average quality costs \$25 and a high-end sweater costs \$30. Now the same money buys less valuable product, as if their wage has gone down!
	\2 Americans could've had a sweater and \$5. Now they just have a sweater. Broken window!

\1 All other American producers are hurt.
	\2 That \$5 could have been spent directly on American services.
	\2 Even if it were also spent in Britain, it is bad for other American producers. Consumers don't spend in Britain, so Britains don't receive money to spend in America, so export suffers.
	
\1 Tariffs also negate the gains made from better transportation. 
	\2 The transportation industry tries to make shipping \$1 cheaper...
	\2 ...and then a \$2 tariff is imposed.

\1 Tariffs cause a loss in productivity.
	\2 We \emph{know} American sweater companies are not the best in the world; the British do it better.
	\2 Imposing a tariff to create jobs in inefficient industry means those people aren't employed in efficient industries where America \emph{is} the best. American labour is now less efficient.

%\1 Tariffs are like taxes, except even more wasteful.
%	\2 From the POV of an American citizen, a tariff is like a tax on British products used to subsidise the new American industry.
%	\2 From the POV of a British producer, a tariff is an invisible wall. They don't see any of that money, but it still hampers their export.
%	\2 From the POV of the new American industry, they just charge \$30 with an extra \$5 overhead versus the British, which is footed by the consumer.

\1 In summary: were jobs "created" with a tariff? Does repealing it "destroy" jobs?
	\2 Every employee's wage became less valuable due to the tariff.
	\2 Jobs were lost in efficient industries due to the tariff.
	\2 Without the tariff, more revenue flows to all domestic producers directly or indirectly. Either way, they can employ more people.

\1 Nevertheless, we shouldn't deny that repealing an existing tariff has short-term losses:
	\2 The people working in the industry become unemployed, and possibly need to learn new skills.
\end{outline}

\section{The Drive For Exports}
\begin{outline}
\1 Foreign exchange 101: currency cannot appear or disappear.
	\2 If a European uses PayPal to buy an American product in dollars, what happens (behind the scenes) is that his euros are traded with someone who wants euros in exchange for dollars, and those dollars are paid to the American.
		\3 To the customer, it looks like the euro vanished and turned into a dollar.
		\3 In actuality, his euros are now with the trader, since they can't vanish.
	\2 Without instant-trade technology, there are two ways the transaction can happen:
		\3 European has no dollars: he pays in euros, and now the American has euros. The American can use these to buy European products, or can exchange them with another trader.
		\3 European has dollars: he pays in dollars, which the American can use domestically.

\1 You can't grow your economy by giving/loaning money to foreign countries to buy your products.
	\2 This is just redistributionism with extra steps: you tax citizens, give Africa dollars, and then they buy certain American products with those tax dollars.
	\2 Worse, they can default on the loan.
		\3 The fallacy is that "the part of the loan they didn't default on helped our export", but that's a lie, because it was our money to begin with.

\1 The real gain of foreign trade is not export, but import.
	\2 You export to China because you want to receive their currency to then buy unique/more products from them than you could have bought with the same value in your own country.
\end{outline}