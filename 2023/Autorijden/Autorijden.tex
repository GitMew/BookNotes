% !TeX spellcheck = nl_NL
\chapter{Autorijden van A tot Z}{Flor Koninckx}
Een kort overzicht van minst intuïtieve, te memoriseren concepten in het verkeer.

\section{Examens}
\begin{outline}
\1 Theorie:
	\2 Lichte fout: 1 punt af. Zware fout: 5 punten af.
	\2 Gebuisd bij $\geq 10$ punten af.
	\2 Indien zelfstudie, na 2 keer buizen verplicht 12 uur les volgen.

\1 Praktijk:
	\2 Videotest, onderhoudstest, rijtest.
\end{outline}

\section{Verkeersreglement}
\begin{outline}
\1 Terminologie van de openbare weg:
	\2 \emph{Weg}: ruimte voor elk soort verplaatsing.
	\2 \emph{Rijbaan}: deel van de weg voor voertuigen.
	\2 \emph{Fietspad}: deel van de weg voor fietsers, vaak op de rijbaan.
	\2 \emph{Berm}: deel naast de rijbaan voor parkeren en/of voetverkeer.
	\2 \emph{Voetpad}: deel van de weg voor voetgangers.
	\2 Speciale soorten weg:
		\3 \emph{Straat}: soort weg met extra functies ernaast, bv.\ wonen.
			\4 \emph{Bebouwde kom}: netwerk van straten.
		\3 \emph{Plein}: verharde open vlakte waar alle rijrichtingen toegestaan zijn.
		\3 \emph{Rond punt}: eenrichtingsweg die in een cirkel loopt
			\4 \emph{Rotonde}: rond punt met voorrang voor de voertuigen erop i.p.v.\ eraf.
			
\1 Terminologie van weggebruikers:
	\2 \emph{Voetganger}: wandelaar, loper, kruiwagenbestuurder, rolschaatser, etc ...
		\3 Een fiets of brommer duwen telt niet als ermee rijden.
		\3 Een motorfiets of auto duwen telt \emph{wel} als ermee rijden.
	\2 \emph{Voertuig}: ding waarmee mensen of goederen verplaatst worden.
		\3 \emph{Voortbewegingstoestel}: heeft geen pedalen en geen of zwakke motor (25 km/u).
		\3 \emph{Rijwiel}: heeft pedalen en geen of verwaarloosbare motor.
		\3 \emph{Motorvoertuig}: heeft een motor.
	\2 \emph{Bromfiets}: $V_\text{cilinder} \leq \SI{50}{cm^3}  \;\land\;  P_\text{motor} \leq \SI{4}{kW}$, en een van onderstaande:
		\3 Klasse A: $v_\text{max} \leq \SI{25}{km/u}$
		\3 Klasse B: $v_\text{max} \leq \SI{45}{km/u}$
			\4 Inclusief "speed pedelec", een elektrische bromfiets met nutteloze pedalen.
	
	\2 \emph{Motorfiets}: geen bromfiets, en een van onderstaande:
		\3 \emph{Moto}:  $\#\text{wielen} = 2$
		\3 \emph{Trike}: $\#\text{wielen} = 3  \;\land\;  m_\text{eigen} \leq \SI{1000}{kg}$
		\3 \emph{Quad}: $\#\text{wielen} = 4  \;\land\;  m_\text{eigen} \leq \SI{400}{kg} \;\land\; P_\text{motor} \leq \SI{15}{kW}$

	\2 \emph{Auto}: alle voertuigen die resteren.

\1 Rijbewijzen en voertuigen:
\begin{center}
\begin{tabular}{llllll}
Voertuig           & Helm & Nummerplaat & Rijbewijs & Minimumleeftijd & Passagiers \\ \hline
Fiets 			   & nee & nee & nee & 0 & nee \\
Bromfiets klasse A & ja & ja & / & 16 & vanaf 18 \\
Bromfiets klasse B & ja & ja & AM & 16 & vanaf 18 \\
Motorfiets         & ja & ja & A & 18-21 & enkel met helm \\
Auto (MTM $\leq$ 3.5t) 			   & nee & ja & B & 18 & $\leq$8 
\end{tabular}
\end{center}

\1 Terminologie over massa:
	\begin{equation}\label{eq:mass}
		\underbrace{m_\text{eigen}}_\text{voertuigclassificatie} \leq \underbrace{m_\text{beladen}\,}_\text{verkeersborden\footnotemark} \leq \underbrace{\,\text{MTM}\,}_\text{rijbewijzen}
	\end{equation}
	\footnotetext{Uitzondering: onderborden bij borden over \textsl{parkeren, inhalen, cruise control, snelheid (PICS)} hebben betrekking op MTM, niet beladen massa.}
	\2 \emph{Eigen massa}: massa met volle tank, geen inzittenden en geen goederen.
	\2 \emph{Beladen massa}: massa tijdens het rijden.
	\2 \emph{Maximaal toegelaten massa (MTM)}: uiterste beladen massa volgens de fabrikant.
	
\1 Slepen:
	\2 \emph{Sleep}: voertuig dat iets sleept.
	\2 Gesleepte dingen zijn:
		\3 \emph{Aanhangwagen}: ding gemaakt om gesleept te worden. 
		\3 Defecte auto: mag niet op de autosnelweg, en met maximaal 25 km/u.

\1 Belangrijke getallen:
	\2 Voetgangers die $\leq$20 m van een zebrapad staan, mogen alleen daarover oversteken.
	\2 Kinderen die $\leq$10 jaar zijn, mogen op het voetpad rijden met rijwielen.
	\2 Aantal wegkapiteins voor groepen van helmdragers:
\begin{center}
\begin{tabular}{ll}
Groepsgrootte & Kapiteins \\ \hline
0-15 & 0 \\
15-50 & 0-2 \\
50-$\infty$ & 2
\end{tabular}
\end{center}

\1 Bevelen:
	\2 Verkeerspiramide:
	\begin{equation}
		\text{regels} < \text{markeringen} < \text{borden} < \text{lichten} < \text{agenten}\vspace{-1em}
	\end{equation}
	\2 Agent:
		\3 Hand in de lucht: iedereen stopt.
		\3 T-pose dwars met uw voertuig: stop.
		\3 T-pose parallel met uw voertuig: rij door, en sla indien nodig af achter de agent.
	\2 Prioritaire voertuigen: voorrang bij zwaailicht \emph{en} sirene.
\end{outline}

\section{Verkeerstekens}
\begin{outline}
\1 Soorten borden:
\begin{center}
\begin{tabular}{r|llll}
	Prefix & Betekenis & Vorm & Randkleur & Vulkleur \\ \hline
	A & gevaar & driehoek & \textcolor{red}{rood} & wit \\
	B & voorrang & (mix) & (mix) & (mix) \\
	C & verbod & cirkel & \textcolor{red}{rood} & wit \\
	D & gebod & cirkel & wit & \textcolor{blue}{blauw} \\
	E & parkeren & cirkel/rechthoek & \textcolor{red}{rood}/geen & \textcolor{blue}{blauw} \\ 
	F & aanwijzing & rechthoek & wit & \textcolor{blue}{blauw}
\end{tabular}
\end{center}

\1 Gevaarsborden:
	\2 Gelden op het punt 150m van het bord.
	\2 Onderborden kunnen dat veranderen naar een andere afstand, en met pijltjes naar een spanne i.p.v.\ een punt.
	
\1 Aanwijzingsborden:
	\2 Hebben weleens de betekenis van gebod $+$ verbod, d.w.z.\ "voor en alleen voor".
	\2 Valkuilen:
		\3 Bebouwde kom versus plaatsaanduiding.
		\3 "N" is normale weg, "A" is autosnelweg, "E" is Europese weg, "R" is ring.

\1 Verbodsborden:
	\2 Gelden tot aan het volgende kruispunt (of tot het volgende gelijkaardige bord).
	\2 Valkuilen:
		\3 Bord voor fiets en bromfiets en moto lijkt op elkaar. Moto toont een mannetje.
		\3 Borden met een auto tellen voor alle voertuigen met $>$2 wielen, \emph{inclusief} moto met zijkar.

\1 Parkeerborden:
	\2 Gelden aan één kant van de weg.
	
\1 Massa-aanduidingen:
	\2 Op een verbodsbord: het kleine getal staat na de komma.
	\2 PICS-regel: zie voetnoot bij \autoref{eq:mass}
\end{outline}

\section{Voertuigen}
Extreem belangrijk hoofdstuk voor het theorie-examen!
\begin{outline}
\1 Belangrijke getallen:
	\2 Afmetingen:
		\3 $\leq$2.55m breed met lading.
		\3 $\leq$4m hoog met lading.
	\2 Ladingen:
		\3 $0$m uitstekend voor de voorste bumper.
		\3 $\leq$1m uitstekend achter de achterbumper.
		\3 $\leq$3m uitstekend achter de achterbumper mits
			\4 de lading ondeelbaar is, en
			\4 wit-rood markeringsbord, 0.5m $\times$ 0.5m, met rood licht achterop en oranje reflectoren erlangs.
	\2 Inzittenden:
		\3 $1$ gordel per persoon.
		\3 $\geq$0.55m zitbreedte voor de chauffeur.
		\3 $\geq$0.4m zitbreedte voor passagiers vooraan, geen achteraan.
	\2 Kinderen:
		\3 $\leq$1.35m groot en $<$18 jaar oud: verplicht in stoeltje en achteraan.
		\3 $\geq$3 jaar oud: toegestaan zonder stoeltje (maar achteraan) mits
			\4 ouders rijden met 3 kinderen en er geen extra stoeltje bij past, of
			\4 niet-ouders rijden over korte afstanden.

\1 Gordel dragen moet niet voor:
	\2 Chauffeurs die achteruitrijden;
	\2 Postbodes tijdens het ronddelen;
	\2 Taxichauffeurs tijdens het vervoeren;
	\2 Politieagenten die een gevaarlijk persoon vervoeren;
	\2 Iedereen met een medisch attest dat gordels optioneel maakt.

\1 Verplicht aanwezig in de auto:
	\2 Documenten (\emph{origineel}, geen kopies):
	\begin{enumerate}
		\item CVO (\emph{certificate of conformity, COC}): identificatie van de wagen als toestel, en bewijs van de fabrikant dat het alle normen naleeft.
		\item Verzekeringsbewijs (\emph{insurance}): bewijs dat uw stommiteiten aan slachtoffers zullen worden terugbetaald.
		\item Kentekenbewijs (\emph{registration}): link tussen nummerplaat en CVO.
		\item Keuringsbewijs: na 4 jaar moet het voertuig elke 10 à 12 maanden (niet meer, niet minder) in de garage gecontroleerd worden.
	\end{enumerate}
	
	\newpage

	\2 Objecten:
	\begin{enumerate}
		\item Fluohesje
		\item Gevarendriehoek
		\item Brandblusser
		\item EHBO-koffer
	\end{enumerate}

\1 Autolichten:
	\2 Kleuren:
		\3 Wit: auto beweegt naar de kijker.
		\3 Rood: auto beweegt weg van de kijker.
		\3 Oranje: gevaar (bv.\ een manoeuvre).
	\2 Soorten:\symfootnote{Met sterretje $=$ niet verplicht, wel toegelaten.}
		\3 Voorwaarts:
			\4 \emph{Kruislichten} ("low beams"): neerwaartse, gemiddelde lichten.
			\4 \emph{Grootlichten} ("faren", "high beams"): voorwaartse, felle lichten.
		\3 Achterwaarts:
			\4 \emph{Achterlichten}: continue lichten, brandend samen met kruislichten.
			\4 \emph{Remlichten}: twee (of drie*) lichten die door het rempedaal bediend worden.\footnote{Niet gewoon "bij het remmen" vanwege het bestaan van "engine braking".}
			\4 \emph{Plaatlicht}: lichtjes rondom de nummerplaat.
			\4 \emph{Achteruitrijlicht}:* wit licht dat brandt bij het schakelen in achteruit.
		\3 Beide:
			\4 \emph{Standlichten}: zwakke lichten om geparkeerd voertuig aan te geven.
			\4 \emph{Mistlichten}: felle maar zeer directionele lichten.
			\begin{itemize}
				\item Vooraan:* laag tegen de weg.
				\item Achteraan: twee onder de achterlichten, of één in het midden.
			\end{itemize}
			\4 \emph{Richtingsaanwijzers}: pinkende oranje lichten vooraan en achteraan.

\1 Wanneer lichten aan, wanneer lichten uit:
\begin{center}
\begin{tabular}{r|p{6cm}|p{6cm}}
	\bfseries Licht & \bfseries Verplicht aan & \bfseries Verplicht uit \\\hline\hline
	Standlichten    & & $\geq$100m zicht \\\hdashline
	Kruislichten    & Tussen zonsondergang en -opgang, of bij $\leq$200m zicht. &   \\\hdashline
	Grootlichten    & & $\geq$100m zicht, of bij een voorligger op 50m, of bij een naderende tegenligger, of bij lichtclaxon van een verre tegenligger.   \\\hdashline
	Mistlichten (rood) & Regen met opspattend water, of bij $\leq$100m zicht in mist of sneeuw. & alle andere situaties \\\hdashline
	Mistlichten (wit)  &  & wanneer het rode verplicht uit moet \\\hdashline
	4 pinkers       & & Alle situaties die geen direct gevaar vormen voor anderen, bv.\ als excuus om enkele minuten fout te parkeren. Mag wel bij pech, om dieren op de weg te signaleren, als een tunnel stilligt, als er kinderen uit een bus afstappen, ...  \\\hdashline
	Oranje zwaailicht & Landbouwvoertuigen op een weg met $\geq$2 rijstroken, of wanneer de kruislichten verplicht aan moeten. & 
\end{tabular}
\end{center}
	\2 Bij parkeren/stilstaan:
		\3 Mistlichten moeten uit: alle lichten verplicht uit, behalve standlichten. 
		\3 Mistlichten moeten aan: grootlichten verplicht uit.

\1 Andere regels:
	\2 Mensen mogen alleen in een sleep zitten, dus \emph{niet} in een gesleepte caravan.
	\2 Airbags werken enkel samen met gordels; voor babystoeltjes moet de airbag afstaan!
	\2 Nummerplaat en lichten moeten altijd zichtbaar of verlengd zijn.
\end{outline}

\section{Plaats op de openbare weg}
\begin{outline}
\1 Busstroken mogen alleen bereden worden om af te slaan bij het volgende kruispunt.
\1 1m plaats houden in de bebouwde kom:
	\2 langs obstakels, als voetpad voor voetgangers;
	\2 langs fietsers, voor beweegruimte.
	
\1 Bij oversteekplaatsen:
	\2 Zebrapad: wachtende en overstekende voetgangers hebben voorrang.
	\2 Fietsoversteekplaats: \emph{enkel} overstekende fietsers hebben voorrang!

\1 Fietspad vs.\ fietssuggestiestrook:
	\2 Fietspad: aangeduid met lijnen en borden. 
		\3 Verplicht voor fietsers en klasse A, verboden voor de rest.
		\3 Verboden voor klasse A indien gemengd met voetgangers (zonder scheiding).
	\2 Fietssuggestiestrook: aangeduid met gekleurd asfalt. Geen regels.
\end{outline}

\section{Voorrang}
\todo{95-106}

\section{Kruisen en inhalen}
\begin{outline}
\1 Zonder markeringen/borden is inhalen verboden als en slechts als het gevaar voor anderen bezorgd.
	\2 Voornamelijk bij beperkte zichtbaarheid met tegenliggers (slecht weer, in een bocht, voor een heuveltop ...).
	\2 Iemand die ingehaald wordt, mag vandaar ook niet versnellen.
\1 Tripleren mag als er drie rijstroken zijn.
\1 Voorsorteren telt niet als inhalen.
\end{outline}

\section{Snelheid}
\begin{outline}
\1 Belangrijke getallen:
	\2 20 km/u: maximale snelheid in een woonerf (bord met een huis op een slee).
	\2 30 km/u: maximale snelheid in een fietsstraat, landbouwstraat, en over een verkeersdrempel.
	\2 Bebouwde kom:
		\3 30 km/u: default maximale snelheid in de bebouwde kom in Brussel.
		\3 50 km/u: default maximale snelheid in de bebouwde kom in Vlaanderen en Wallonië.
	\2 Buiten bebouwde kom:
		\3 Snelweg:
			\4 70 km/u: minimale snelheid op de snelweg.
			\4 120 km/u: default maximale snelheid op de snelweg.
		\3 Elders:
			\4 70 km/u: default maximale snelheid in Vlaanderen en Brussel.
			\4 90 km/u: default maximale snelheid in Wallonië.
			\4 120 km/u: default maximale snelheid op een weg met 2 richtingen gescheiden door een stenen/stalen verhoging en telkens $\geq$2 rijvakken.
	\2 Overtredingen:
		\3 Intrekking van rijbewijs:
			\4 20 km/u te snel in een bebouwde kom of woonerf.
			\4 30 km/u te snel op een autoweg of snelweg.
		\3 Rijverbod:
			\4 30 km/u te snel in een bebouwde kom of woonerf.
			\4 40 km/u te snel op een autoweg of snelweg.

\1 Afstanden:
	\2 Reactieafstand, remafstand, stopafstand:
	\begin{equation}
		\Delta x_\textrm{stop} = \Delta x_\textrm{reactie} + \Delta x_\textrm{rem}
	\end{equation}
	
	\2 Heuristieken (in meter):
		\3 Reactieafstand:
			\4 Formule: neem $1$ seconde reactietijd, dan is
			\begin{equation}
				\Delta x_\textrm{reactie} = \frac{v_\textrm{kph}}{3.6} \approx \frac{3}{10}\, v_\textrm{kph}
			\end{equation}
			\4 Om niet te botsen tegen mensen met dezelfde snelheid: twee seconden tussen.
		\3 Remafstand:
			\4 Formule: neem constante deceleratie $a = -\mu g$, dan is
			\begin{equation}
				\Delta x_\textrm{rem} = \frac{1}{2} \left( \frac{v_\textrm{kph}}{3.6 \sqrt{\mu g}} \right)^{\!2} \approx \frac{1}{2} \left( \frac{v_\textrm{kph}}{10} \right)^{\!2}
			\end{equation}
			voor droog wegdek. Vermenigvuldig met 1.5 (tel helft erbij) voor nat wegdek.

		\2 Stopafstand:
			\3 Te kennen tabel voor droge weg:
			\begin{center}
			\begin{tabular}{rr}
				$v_\textrm{kph}$ & $\Delta x_\textrm{stop}$ \\\hline
				50 & 30 \\
				70 & 45 \\
				90 & 70 \\
				110 & 95 \\
			\end{tabular}
			\end{center}
			
			\3 Formule accuraat tot op 5 meter: $\Delta x_\textrm{stop} = v_\textrm{kph} - 20$.
\end{outline}

\section{Auto(snel)wegen}
\begin{outline}
\1 Wat maakt auto(snel)wegen speciaal? Gemotoriseerd eenrichtingsverkeer.
	\2 Verbod op parkeren en stilstaan.
	\2 Verbod op achteruitrijden.
	\2 Verbod op omkeren (zelfs via de gaten in de berm tussen de banen).
	\2 Verbod op voetgangers, fietsers, ruiters, tractors, brommers, quads, ...

\1 Wat maakt een autosnelweg speciaal?
	\2 Wegen snijden, maar kruisen niet.
	\2 Minimumsnelheid.
	
\1 Wat te doen bij pech of ongeval?
\begin{enumerate}
	\item Rij pechstrook op. (Opgelet: spitsstrook is \emph{nooit} een pechstrook.)
	\item Iedereen stapt uit en gaat achter de vangrail staan.
	\item Bestuurder doet fluovest aan.
	\item Bestuurder zet gevarendriehoek 30 m (autoweg) of 100 m (autosnelweg) ervoor.
\end{enumerate}
\end{outline}

\section{Speciale zones}
\begin{outline}
\1 Overweg: vertragen en stoppen indien \emph{minstens} één signaal gegeven wordt.
	\2 Trein zichtbaar.
	\2 Bel luidt.
	\2 Slagbomen bewegen (neer of op).
	\2 Lichten pinken rood.
\1 Verkeersdrempel: vertragen tot $\leq$30 km/u.

\1 Plaatsen waar auto's mogen rijden, maar zich submissief moeten gedragen t.o.v.\ anderen:
	\2 Erf. Spelen toegestaan (d.w.z.\ vrij rondlopen op straat).
	\2 Fietsstraat/-zone.
	
\1 Plaatsen waar alleen bewonende auto's mogen rijden, en zo traag mogelijk:
	\2 Speel-/schoolstraat. Spelen toegestaan.

\1 Plaatsen waar auto's niet mogen rijden:
	\2 Landbouwweg, d.w.z.\ weg voor voetgangers-fietsers-ruiters-tractors.
	\2 Voetgangerszone.
	
\1 Tunnels tellen als stinkende, donkere autowegen.
	\2 Best alle ramen dicht en lucht laten circuleren.
	\2 Zonnebril op voor de tunnel en af in de tunnel.
	\2 Indien totale stilstand door file: 4 pinkers op en motor af.
\end{outline}

\section{Stilstaan en parkeren}
\begin{outline}
\1 Stilstaan hangt af van het doel, niet de tijd.
	\2 Urenlang laden en lossen telt als stilstaan.
	\2 Enkele seconden stoppen om de kaart te checken telt als parkeren.
	\2 Stilstaan omvat niet "gestopt zijn" (remmen) en "geïmmobiliseerd zijn" (auto kapot).

\1 Belangrijke getallen:
	\2 $\geq$1.5m berm laten voor voetgangers bij parkeren.
	\2 $\geq$1.65m voertuighoogte $\Rightarrow$ $\geq$20m afstand tot verkeerslicht of -bord bij parkeren/stilstaan.
	\2 $\geq$5m afstand tussen \emph{neus} van auto en oversteekplaats of oprit-/afrit voor fietsers bij parkeren/stilstaan.
	\2 $\geq$5m vs.\ $\geq$20m afstand tot hoek van kruispunt zonder vs.\ met lichten bij parkeren.
	\2 $\geq$1m afstand tot volgende auto bij stilstaan.
	\2 $\geq$15m afstand tot bushaltebord bij stilstaan.
	\2 $\leq$3m rijbaanbreedte na stilstaan $\Rightarrow$ parkeerverbod.

\1 Ronde borden stoppen met gelden bij het volgende kruispunt.
	\2 Dat geldt voor snelheidsborden, parkeerborden, ...

\1 "Beurtelings parkeren in bebouwde kom" is op basis van huisnummers. Eerste helft van de maand zijn de oneven huisnummers.
\end{outline}

\section{Bus en tram}
\begin{outline}
\1 Tram:
	\2 Heeft altijd voorrang.
	\2 Kan op twee plekken rijden:
		\3 Eigen berm, bv.\ met gras. Niet toegelaten voor anderen.
		\3 Gewone weg. Niet toegelaten indien afgebakend door witte strepen (bijzondere ...).
	
\1 Bus:
	\2 Heeft alleen speciale voorrang bij het verlaten van een bushalte binnen bebouwde kom.
	\2 Zonder verdere aanduiding is een busstrook voor bussen, en auto's die gaan inslaan.
\end{outline}

\section{Gedrag jegens anderen}
\begin{outline}
\1 Voetgangers: 
	\2 Bij aanstalten van oversteken: stoppen.
	\2 Als het al rood is en ze zijn nog aan het oversteken: stoppen.
	\2 Bij stilstaan/parkeren: zorg dat je hen niet verder op straat dwingt dan nodig.
	
\1 Fietsers:
	\2 1m resp.\ 1.5m afstand bij voorbijsteken binnen resp.\ buiten bebouwde kom.

\1 Konvooien (leger, ruiters, motards, fietsers, schoolkinderen ...):
	\2 Hebben het recht op leider met C3-bordje die verkeer regelt.
	\2 Hebben het recht om niet gescheiden te worden door anderen.
\end{outline}

\section{Alcohol en drugs}
\begin{outline}
\1 Alcohol:
	\2 Ademlucht:
		\3 Eerst draagbare test. Indien alarm: analysemachine (wettelijk bewijs).
		\3 Limieten: 0.22 mg/l (alarm) en 0.35 mg/l (positief) lucht.
	\2 Bloed:
		\3 Niet verplicht. Alleen gebruikt om ademanalyse te disputeren.
		\3 Limieten: 0.5 g/l (alarm) en 0.8 g/l (positief) bloed
	\2 Straffen:
		\3 Alarm: 3 uur rijbewijs kwijt $+$ boete.
		\3 Positief: 6 uur rijbewijs kwijt $+$ boete $+$ gerechtelijke procedure.
		
\1 Andere drugs:
	\2 Eerst soberheidstest. Indien positief: speekseltest.
	\2 Straffen: 12 uur rijbewijs kwijt $+$ boete $+$ gerechtelijke procedure.
	
\1 Algemeen:
	\2 Iemand aanzetten tot dronken worden of ermee meerijden of aanzetten tot rijden, is medeplichtig.
	\2 Verzekering covert de kosten van een ongeval onder invloed niet.
	\2 Politie mag verplichten tot test aan iedereen die een voertuig bestuurt, daar aanstalten toe maakt, of het rijden begeleidt.
		\3 Weigeren telt automatisch als positief (0.35 mg/l).
		\3 Eenmalig mag men 15 minuten wachttijd inroepen, alsook een tweede test.
		\3 Indien niet aan het besturen, is gerechtelijke procedure onmogelijk. Enkel boete.
	\2 Recidivisme heeft zware straffen, o.a.\ verplicht rijexamen.
\end{outline}

\section{Ongevallen}
\begin{outline}
\1 Geen lichamelijke schade: vul samen Europees aanrijdingsformulier in.

\1 Wel lichamelijke schade: voor wie kan,
\begin{enumerate}
	\item \emph{Voorkomen}: zet knipperlichten op en ruim de baan om verdere accidenten te voorkomen.
	\item \emph{Brand}: indien slachtoffers in of naast brandende auto, blus de brand.
	\item \emph{Slachtoffers}: breng in veiligheid indien echt nodig, check bewustzijn, adem, hartslag.
	\item \emph{Hulpdiensten}: bel 112 voor politie en mogelijks ambulance.
	\item Papierwerk is voor later.
\end{enumerate}

\end{outline}

\section{Techniek}
\begin{outline}
\1 Algemeen:
	\2 Auto alleen servicen als hij koud is (maar niet bevroren).
	\2 Altijd zorgen dat hij waterpas staat om correcte peilen af te lezen.

\1 Remvloeistof:
	\2 Brengt hydraulische druk van rempedaal naar remblokken.
	\2 Peil checken: min/max op het reservoir zelf (met gele sticker \!\tikz[scale=0.15,baseline=-0.3em]{\draw[delta angle=90] (0,0) circle[radius=1] (-45:1.35) arc[start angle=-45, radius=1.35] (135:1.35) arc[start angle=135, radius=1.35];}, zie \href{https://www.holtsauto.com/prestone/news/understanding-brake-fluid-101/}{afbeeldingen}).
	\2 Niet zelf bijvullen, want kan op grotere problemen wijzen.

\1 Motorolie:
	\2 Zorgt dat de bewegende onderdelen van de motor gesmeerd blijven. Verdampt door hitte.
	\2 Peil checken: met de gele peilstok (zie \href{https://www.youtube.com/watch?v=t_vq6Z8PKZ8}{video}).	
	\2 Bijvullen (niet in hetzelfde gat) tot peilstok blij is.

\1 Koelvloeistof (bevat antivries):
	\2 Vervoert warmte van motor naar radiator.
	\2 Peil checken: op reservoir zelf (doorschijnende plastic bidon).
	\2 Bijvullen tot aangegeven maximum. Moet nog kunnen uitzetten!

\1 Banden:
	\2 Spanning:
		\3 Correcte spanning bepaald door de fabrikant.
		\3 Voor zware ladingen best 0.2 bar bijpompen.
	\2 Groeven:
		\3 $\geq$1.6 mm groefdiepte verplicht. Nieuwe banden hebben $\approx$9 mm.
		\3 In de winter:
			\4 Winterbanden hebben agressievere groeven.
			\4 Spijkerbanden hebben metalen toppen flush met het rubber. \\ Beperkingen: alleen in de winter. max 90 km/u op snelweg en 60 km/u elders.
			\4 Sneeuwkettingen hebben letterlijk kettingen rond de banden. \\ Beperkingen: alleen op sneeuw of ijzel.

\1 Meer over remmen:
	\2 ABS (anti-lock breaking system): onderbreekt het remmen in kleine periodes opdat de wielen toch bestuurbaar blijven (zie \href{https://www.youtube.com/watch?v=mlLYJW-yIIg}{video}).
	\2 ESP (electronic stability program): programma dat beslist naar welke wielen de remmen gaan opdat de auto niet onbedoeld uitwijkt (zie \href{https://www.carhelper.ch/blog/2018/01/23/abs-and-esp-what-is-the-difference/}{artikel}).
\end{outline}

\section{Defensief rijgedrag}
\begin{outline}
\1 Defensief rijden (t.o.v.\ agressief) betekent:
	\2 Zelf geen fouten maken;
	\2 Fouten van anderen anticiperen, opmerken, en tolereren.

\1 Kijkgedrag:
	\2 Niet afgeleid;
	\2 Naar alle tekens;
	\2 Naar alle weggebruikers;
	\2 Ver vooruit.

\1 Bij een kruispunt:
\begin{enumerate}
	\item Kijk achteruit.
	\item Rem.
	\item Kijk naar het licht/bord.
	\item Kijk of de doelbaan aan de overkant vrij is (áchter alle oversteekplaatsen).
	\item Kijk links (nabij gevaar).
	\item Kijk rechts (toekomstig gevaar).
	\item Rij het kruispunt op.
	\item Rij tegenliggers voorbij.
	\item Draai de wielen daarna pas en geef gas.
\end{enumerate}
\end{outline}

\section{Appendix: Extra regeltjes}
Verzameld op basis van \href{https://mijnrijbewijs.eu/}{deze website}.
\begin{outline}
\1 De bestuurder is niet verantwoordelijk voor de gordel van inzittenden.
\1 Lichten:
	\2 Grootlichten moeten worden gedoofd op 50m van een andere persoon.
	\2 De kleuren en symbolen voor alle soorten licht.
\1 Boetes tellen ook als straf.
\1 De alcohollimiet voor taxichauffeurs is niet 0.22/0.5 maar wel 0.09/0.2.
\end{outline}